%%
%% Automatically generated file from DocOnce source
%% (https://github.com/hplgit/doconce/)
%%

% #define PREAMBLE

% #ifdef PREAMBLE
%-------------------- begin preamble ----------------------

\documentclass[%
oneside,                 % oneside: electronic viewing, twoside: printing
final,                   % draft: marks overfull hboxes, figures with paths
10pt]{article}

\listfiles               % print all files needed to compile this document

\usepackage{relsize,makeidx,color,setspace,amsmath,amsfonts,amssymb}
\usepackage[table]{xcolor}
\usepackage{bm,microtype}

\usepackage[pdftex]{graphicx}

\usepackage[T1]{fontenc}
%\usepackage[latin1]{inputenc}
\usepackage{ucs}
\usepackage[utf8x]{inputenc}

\usepackage{lmodern}         % Latin Modern fonts derived from Computer Modern

% Hyperlinks in PDF:
\definecolor{linkcolor}{rgb}{0,0,0.4}
\usepackage{hyperref}
\hypersetup{
    breaklinks=true,
    colorlinks=true,
    linkcolor=linkcolor,
    urlcolor=linkcolor,
    citecolor=black,
    filecolor=black,
    %filecolor=blue,
    pdfmenubar=true,
    pdftoolbar=true,
    bookmarksdepth=3   % Uncomment (and tweak) for PDF bookmarks with more levels than the TOC
    }
%\hyperbaseurl{}   % hyperlinks are relative to this root

\setcounter{tocdepth}{2}  % number chapter, section, subsection

\usepackage[framemethod=TikZ]{mdframed}

% --- begin definitions of admonition environments ---

% --- end of definitions of admonition environments ---

% prevent orhpans and widows
\clubpenalty = 10000
\widowpenalty = 10000

% --- end of standard preamble for documents ---


% insert custom LaTeX commands...

\raggedbottom
\makeindex

%-------------------- end preamble ----------------------

\begin{document}

% endif for #ifdef PREAMBLE
% #endif


% ------------------- main content ----------------------



% ----------------- title -------------------------

\thispagestyle{empty}

\begin{center}
{\LARGE\bf
\begin{spacing}{1.25}
Master program in Computational Physics, Mathematics and Life Science
\end{spacing}
}
\end{center}

% ----------------- author(s) -------------------------

\begin{center}
{\bf Andreas Austeng${}^{1}$} \\ [0mm]
\end{center}


\begin{center}
{\bf Michele Cascella${}^{2}$} \\ [0mm]
\end{center}


\begin{center}
{\bf Marianne Fyhn${}^{3}$} \\ [0mm]
\end{center}


\begin{center}
{\bf Morten Hjorth-Jensen${}^{4}$} \\ [0mm]
\end{center}


\begin{center}
{\bf Hans Petter Langtangen${}^{1, 5}$} \\ [0mm]
\end{center}


\begin{center}
{\bf Anders Malthe-Sørenssen${}^{4}$} \\ [0mm]
\end{center}


\begin{center}
{\bf Knut Mørken${}^{6}$} \\ [0mm]
\end{center}


\begin{center}
{\bf Joakim Sundnes${}^{1, 5}$} \\ [0mm]
\end{center}

\begin{center}
% List of all institutions:
\centerline{{\small ${}^1$Department of Informatics, University of Oslo}}
\centerline{{\small ${}^2$Department of Chemistry, University of Oslo}}
\centerline{{\small ${}^3$Department of Biosciences, University of Oslo}}
\centerline{{\small ${}^4$Department of Physics, University of Oslo}}
\centerline{{\small ${}^5$Simula Research Laboratory}}
\centerline{{\small ${}^6$Department of Mathematics, University of Oslo}}
\end{center}
    
% ----------------- end author(s) -------------------------

% --- begin date ---
\begin{center}
October 2015
\end{center}
% --- end date ---

\vspace{1cm}


% !split
\subsection*{Master program in Computational Physics, Mathematics and Life Science}

% --- begin paragraph admon ---
\paragraph{}
We propose a new Master of Science program at the Faculty of Mathematics and Natural Sciences of the University of Oslo. This program is called  \textbf{Computational Physics, Mathematics and Life Science}, with acronym  \textbf{CS}


The program is a collaboration between five departments and classical disciplines:

\begin{itemize}
 \item Department of Biosciences

 \item Department of Chemistry

 \item Department of Informatics

 \item Department of Mathematics

 \item Department of Physics
\end{itemize}

\noindent
The program will be administrated by the Department of Physics.
The program is based on the highly successful Computational Physics direction under the present Master program
in Physics at the University of Oslo.

The program is multidisciplinary and all students who have completed
undergraduate studies in science and engineering, with a sufficient
quantitative background, are eligible.  The language of instruction is
English.
% --- end paragraph admon ---



% !split
\subsection*{Strategic importance}

The program will educate the next generation of cross-disciplinary
science students with the knowledge, skills, and values needed to pose
and solve current and new scientific, technological and societal
challenges. The program will lay the foundation for cross-disciplinary
educational, research and innovation activities at the Faculty. The
program will contribute to building a common cross-disciplinary
approach to the key strategic initiatives at the Faculty: Energy,
Materials, Life Science, and Enabling Technologies.

A specific aim of this program is to develop the students' ability to pose and
solve problems that combine physical insights with mathematical tools
and computational skills. This provides a unique combination
of applied and theoretical knowledge and skills. These features are invaluable
for the development of multi-disciplinary educational and research programs.
The main focus is not to educate computer
specialists, but to educate students with a solid understanding in basic science
as well as an integrated knowledge on how  to use
essential methods from computational science. This requires an
education that covers both the specific disciplines like physics, biology,
geoscience, mathematics etc with a strong background in computational science.

To build up an activity in computational life science would naturally bring in colleagues from the Norwegian University of Life Sciences,
adding to this axis their year long experiences.
The program could thus be a dual collaboration between the University of Oslo and the Norwegian University of Life Sciences, with eventual
double Master of Science degree agreements.

% !split
\subsection*{Scientific and educational motivation}


% --- begin paragraph admon ---
\paragraph{Applications of simulation.}
Numerical simulations of various systems in science are central to our
basic understanding of nature and technlogy.
The increase in computational power,
improved algorithms for solving problems in science as well as access
to high-performance facilities, allow researchers nowadays to study
complicated systems across many length and energy scales. Applications
span from studying quantum physical systems in nanotechnology and the
characteristics of new materials or subamotic physics at its smallest
length scale, to simulating galaxies and the evolution of the universe.
In between, simulations are key to understanding
cancer treatment and how the brain works,
predicting climate changes and this week's weather,
simulating natural disasters, semi-conductor devices,
quantum computers, as well as assessing risk in the insurance and
financial industry. These are just a few topics
already well covered at the University of Oslo and that can be
topics for coming thesis projects as well as research directions.
% --- end paragraph admon ---




% --- begin paragraph admon ---
\paragraph{Job market.}
A large number of the candidates from the five involved departments
get jobs where numerical simulations are central and essential. The proposed
program will raise the educational quality in this area, because
our candidates need a broader understanding of the possibilities
and limitations of computation-based problem solving.
% --- end paragraph admon ---



% !split
\subsection*{Multiscale modeling is a big open research question}


% --- begin paragraph admon ---
\paragraph{}
Today's problems, unlike traditional
science and engineering, involve complex systems with many distinct
physical processes. The wide open research topic of this century, both
in industry and at universities, is how to effectively couple
processes across different length and energy scales. Progress will
rely on a multi-disciplinary approach and therefore a need for
a multi-disciplinary educational program.
% --- end paragraph admon ---




% --- begin paragraph admon ---
\paragraph{}
The proposed program will foster candidates with the right
multi-disciplinary background and computational thinking for
understanding today's simulation technology and its challenges.
% --- end paragraph admon ---




% !split
\subsection*{The new program combines old and new initiatives}

% --- begin paragraph admon ---
\paragraph{}

This program builds on the strengths and successes of two existing Master of Science directions at the University of Oslo, namely the 
programs in Computational Physics (at the Dept.~of Physics) and
Applied Mathematics and Mechanics (at the Dept.~of Mathematics).
These programs were established in 2003.
Based on the experience from these programs, the hope is that the proposed program can enlarge the reach of disciplines where computations play and/or are expected to play  a large. In particular, new directions 
in Computational Life Science need to  be developed to
meet coming needs of the scientific community. We believe this new direction is
best developed in close collaboration with already successful
computational science programs.  In particular, we believe that there are strong synergy effects between the University of Oslo and the Norwegian 
University of Life Sciences in the field of Computational Life Science.
% --- end paragraph admon ---




% !split
\subsection*{Computational Physics at UiO has been a great success}

% --- begin paragraph admon ---
\paragraph{}

This initiative has its roots in the highly successful direction called \href{{http://www.uio.no/english/studies/programmes/physics-master/programme-options/computational/index.html}}{Computational Physics}
under the Master program in Physics at the University of Oslo.

This program has educated almost 60 Master of Science students during
the last ten years.  Over 50\% of these students have continued with
PhD studies in Physics, Chemistry, Mathematics and now recently
Biology connected with the CINPLA project.

The Computational Physics Master program recieved in 2015 the University of Oslo Educational Award.
% --- end paragraph admon ---



% !split
\subsection*{The new program will also host the CSE project}

% --- begin paragraph admon ---
\paragraph{}

The new proposed program will also take a leading responsibility in further
developments of the highly successful \href{{http://www.mn.uio.no/english/about/collaboration/cse/}}{Computing in Science Education} initiative at UiO.  Master of science thesis projects linked up to the CSE project will be offered.

If the program becomes successful, it will naturally lead to
new cross-disciplinary research and a need for a new department
in computational science.
% --- end paragraph admon ---




% !split
\subsection*{Computing competence}

% --- begin paragraph admon ---
\paragraph{}
Computing means solving scientific problems using computers. It covers
numerical as well as symbolic computing. Computing is also about
developing an understanding of the scientific process by enhancing
algorithmic thinking when solving problems.  Computing competence has
always been a central part of the science and engineering
education.

Modern computing competence is about

\begin{itemize}
\item derivation, verification, and implementation of algorithms

\item understanding what can go wrong with algorithms

\item overview of important, known algorithms

\item understanding how algorithms are used to solve mathematical problems

\item reproducible science and ethics

\item algorithmic thinking for gaining deeper insights about scientific problems
\end{itemize}

\noindent
% --- end paragraph admon ---



% !split
\subsection*{Key elements in computing competence}

% --- begin paragraph admon ---
\paragraph{}
The power of the scientific method lies in identifying a given problem
as a special case of an abstract class of problems, identifying
general solution methods for this class of problems, and applying a
general method to the specific problem (applying means, in the case of
computing, calculations by pen and paper, symbolic computing, or
numerical computing by ready-made and/or self-written software). This
generic view on problems and methods is particularly important for
understanding how to apply available, generic software to solve a
particular problem.


Computing competence represents a central element
in scientific problem solving, from basic education and research to
essentially almost all advanced problems in modern
societies. Computing competence is simply central to further
progress. It enlarges the body of tools available to students and
scientists beyond classical tools and allows for a more generic
handling of problems. Focusing on algorithmic aspects results in
deeper insights about scientific problems.

Today's projects in science and industry tend to involve larger teams. Tools for reliable collaboration must therefore be mastered (e.g., version control systems, automated computer experiments for reproducibility, software and method documentation).
% --- end paragraph admon ---




% !split
\subsection*{Learning outcomes}

% --- begin paragraph admon ---
\paragraph{}
Students of this program learn to use the computer as a laboratory for solving problems in science and engineering. The program offers exciting thesis projects from many disciplines: biology and life science, chemistry, mathematics, informatics, physics, geophysics, mechanics, geology, computational finance, digital signal processing and image analysis  – the students choose their  field according to their own interests.

A Master’s degree from this program gives the student   a methodical training in planning, conducting, and reporting large research projects, often together with other students and university teachers.
Projects usually emphasize finding practical solutions,
developing an intuitive understanding of the science and the
scientific methods needed to solve complicated problems, use of many
tools, and not least developing your own creativity and independent
thinking. The thesis work is a scientific project where the students learn to
tackle a scientific problem in a professional manner.
% --- end paragraph admon ---



% !split
\subsection*{Specific skills and general competence}

% --- begin paragraph admon ---
\paragraph{}
The students learn to understand and develop insights in high-level scientific problems, involving a fundamental understanding of the methods and tools which are necessary and to present these results orally and in written form as scientific reports/articles. In addition, students learn to get

\begin{itemize}
\item Deep knowledge of the most fundamental algorithms involved, how to optimize these  and understand statistical uncertainty quantifications.

\item Overview of advanced algorithms and how they can be accessed in available software.

\item Knowledge of high-performance computing elements: memory usage, vectorized and parallel algorithms.

\item Understanding of approximation errors.

\item Knowledge of at least one computer algebra system and how it is applied to perform classical mathematics (calculus, linear algebra, differential equations - with verification).

\item Extensive experience with programming in a high-level language (MATLAB, Python, R). Experience with programming in a compiled language (Fortran, C, C++).

\item Experience with implementing and applying numerical algorithms in reusable software that acknowledges the generic nature of the mathematical algorithms.

\item Experience with debugging software, e.g., as part of implementing comprehensive tests.

\item Experience with programming of testing procedures.

\item Experience with different visualization techniques for different types of computed data.

\item Experience with presenting computed results in scientific reports and oral presentations.

\item Critical evaluation of results and errors

\item Be able to develop software and algorithms for solving complicated scientific problems independently and/or in collaboration with other students.
\end{itemize}

\noindent
By completing a Master of Science thesis you will have developed a critical understanding of the scientific methods you have studied, a better understanding of the scientific process per se as well as having developed perspectives for future work and how to verify and validate your work. A better understanding of ethical aspects of the the scientific method is a central aspect.
% --- end paragraph admon ---




% !split
\subsection*{Structure and courses}

% --- begin paragraph admon ---
\paragraph{}
The table here is an example of a suggested path for a Master of Science project,
with course work the first year and thesis work the last year.


\begin{quote}
\begin{tabular}{llll}
\hline
\multicolumn{1}{l}{  } & \multicolumn{1}{l}{ 10 ECTS } & \multicolumn{1}{l}{ 10 ECTS } & \multicolumn{1}{l}{ 10 ECTS } \\
\hline
4th semester & Master thesis  & Master Thesis  & Master Thesis  \\
\hline
3rd semester & Master thesis  & Master Thesis  & Master Thesis  \\
\hline
2nd semester & Master courses & Master courses & Master courses \\
\hline
1st semester & Master courses & Master courses & Master courses \\
\hline
\end{tabular}
\end{quote}

\noindent
The program is very flexible in its structure and students may opt for starting with their thesis
work from the first semester and scatter the respective course load across all four semesters.
Depending on interests and specializations, there are many courses on computational science which can make
up the required curriculum of course work. Furthermore, courses may be broken up in smaller modules,
avoding thereby the limitation of 10 ECTS per course only. Some of these courses are listed below.
% --- end paragraph admon ---



% !split
\subsection*{Structure and specialized modules}

% --- begin paragraph admon ---
\paragraph{}
The program allows also for replacing regular courses with specialized modules of shorter duration.
These modules will be developed by the program committee but can also be developed in an ad hoc basis
and tailored to the individual projects. Specialized modules can amount to up to the full course requirement of 60 ECTS.



\begin{quote}
\begin{tabular}{llll}
\hline
\multicolumn{1}{l}{  } & \multicolumn{1}{l}{ 10 ECTS } & \multicolumn{1}{l}{ 10 ECTS } & \multicolumn{1}{l}{ 10 ECTS } \\
\hline
4th semester & Master thesis  & Master Thesis  & Master Thesis  \\
\hline
3rd semester & Master thesis  & Master Thesis  & Master Thesis  \\
\hline
2nd semester & Special module & Special module & Special module \\
\hline
1st semester & Special module & Special module & Special module \\
\hline
\end{tabular}
\end{quote}

\noindent
The above set up shows how courses may be broken up in smaller modules.
% --- end paragraph admon ---






% !split
\subsection*{Presently available courses at UiO and NMBU}

% --- begin paragraph admon ---
\paragraph{}
Here follows a list of suggested courses that students may include in their required course load.

\begin{itemize}
\item \href{{http://www.uio.no/studier/emner/matnat/fys/FYS4150/index-eng.html}}{FYS4150 Computational Physics I}

\item \href{{http://www.uio.no/studier/emner/matnat/fys/FYS4411/}}{FYS4411 Computational Physics II}

\item \href{{http://www.uio.no/studier/emner/matnat/fys/FYS4460/}}{FYS4460 Computational Physics III}

\item \href{{http://www.uio.no/studier/emner/matnat/ifi/INF5620/index-eng.html}}{INF5620 Numerical Methods for Partial Differential Equations}

\item \href{{http://www.uio.no/studier/emner/matnat/ifi/INF5631/index-eng.html}}{INF5631 Project on Numerical Methods for Partial Differential Equations}

\item \href{{http://www.nmbu.no/course/FYS388}}{FYS388 Computational Neuroscience}

\item \href{{http://www.uio.no/studier/emner/matnat/math/STK4520/index-eng.html}}{STK4520 Laboratory for Finance and Insurance Mathematics}

\item \href{{http://www.uio.no/studier/emner/matnat/math/STK4021/index-eng.html}}{STK4021 Applied Bayesian Analysis and Numerical Methods}

\item \href{{http://www.uio.no/studier/emner/matnat/math/MAT-INF4130/index-eng.html}}{MAT-INF4130  Numerical Linear Algebra}

\item \href{{http://www.uio.no/studier/emner/matnat/math/MAT-INF4110/index.html}}{MAT-INF4110 Mathematical Optimization}

\item \href{{http://www.uio.no/studier/emner/sv/oekonomi/ECON4240/index.html}}{ECON4240 Equilibrium, welfare and information}

\item \href{{http://www.uio.no/studier/emner/matnat/math/MEK4470/index-eng.html}}{MEK4470  Computational Fluid Mechanics}

\item \href{{http://www.uio.no/studier/emner/matnat/math/MEK4250/index-eng.html}}{MEK4250 Finite Element Methods in Computational Mechanics}
\end{itemize}

\noindent
The program plans to develop other courses in computational science and its applications, ranging from life science to materials science.
Courses on project planning and project administration are also possible to include.
% --- end paragraph admon ---





% !split
\subsection*{Thesis directions}

% --- begin paragraph admon ---
\paragraph{}
The program aims at offering thesis projects in a variety of fields. The scientists involved in this program can offer thesis
topics that cover several disciplines. These are

\begin{itemize}
\item Computational mathematics

\item Computational mechanics and fluid mechanics  (NEED people here)

\item Computational chemistry

\item Computational physics

\item Computational materials science

\item Computational life science

\item Image analysis and signal processing

\item Computational finance and statistics   (NEED people here)
\end{itemize}

\noindent
The thesis projects will be tailored to the student's needs, wishes and scientific background. The projects can easily incorporate topics from more than one discipline.
% --- end paragraph admon ---







% !split
\subsection*{Admission}

% --- begin paragraph admon ---
\paragraph{}
The following higher education entrance qualifications are needed

\begin{itemize}
\item A completed bachelor's degree (undergraduate) comparable to a Norwegian bachelor's degree in one of the following disciplines
\begin{enumerate}

 \item Biology, molecular biology, biochemistry  or any life science degree

 \item Physics, astrophysics, astronomy, geophysics and meteorology

 \item Mathematics, mechanics, statistics and computational mathematics

 \item Computer science and electronics

 \item Chemistry

 \item Materials Science and nanotechnology

 \item Any undergraduate degree in engineering

 \item Mathematical finance and economy

 \item Social economy

\end{enumerate}

\noindent
\item The language of instruction is English. An internationally recognised English language proficiency test is required.
\end{itemize}

\noindent
The above undergraduate degrees have some minimal requirements on specializations which need to be fulfilled.  The average mark
for the specific specialization has at least to be C (letter marks).
As an example, an undergraduate degree in Chemistry has a minimal requirement on chemistry courses. The average mark on these chemistry courses should
at least be C.
% --- end paragraph admon ---



% !split
\subsection*{The program opens up for flexible backgrounds}


% --- begin paragraph admon ---
\paragraph{}
While discipline-based master's programs tend to introduce very strict
requirements to courses, we believe in adapting a computational thesis
topic to the student's background, thereby opening up for
students with a wide range of bachelor's degrees.
A very heterogeneous student community is thought to be a strength and
unique feature of this program.
% --- end paragraph admon ---



% !split
\subsection*{Study abroad and international collaborators}


% --- begin paragraph admon ---
\paragraph{}

Students at the University of Oslo may choose to take parts of
their degrees at a university abroad.

Students in this program have a number of interesting international
exchange possibilities. The involved researchers have extensive
collaborations with other researchers worldwide. These exchange
possibility range from top universities in the USA, Asia and Europe as
well as leading National Laboratories in the USA.
% --- end paragraph admon ---



% !split
\subsection*{Career prospects}


% --- begin paragraph admon ---
\paragraph{}
Candidates who are capable of modeling and understanding complicated
systems in natural science, are in short supply in society.  The
computational methods and approaches to scientific problems students learn
when working on their thesis projects are very similar to the methods
they will use in later stages of their careers.  To handle large
numerical projects demands structured thinking and good analytical
skills and a thorough understanding of the problems to be solved. This
knowledge makes the students unique on the labor market.

Career opportunities are many, from research institutes, universities
and university colleges and a multitude of companies. Examples
include IBM, Hydro, Statoil, and Telenor.  The program gives an
excellent background for further studies, with a PhD as one possible
goal.

The program has also a strong international element which allows students to
gain important experience from international collaborations in
science, with the opportunity to spend parts of the time spent on
thesis work at research institutions abroad.
% --- end paragraph admon ---





% ------------------- end of main content ---------------


% #ifdef PREAMBLE

\printindex

\end{document}
% #endif

