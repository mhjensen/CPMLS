%%
%% Automatically generated file from DocOnce source
%% (https://github.com/hplgit/doconce/)
%%

% #define PREAMBLE

% #ifdef PREAMBLE
%-------------------- begin preamble ----------------------

\documentclass[%
oneside,                 % oneside: electronic viewing, twoside: printing
final,                   % draft: marks overfull hboxes, figures with paths
10pt]{article}

\listfiles               % print all files needed to compile this document

\usepackage{relsize,makeidx,color,setspace,amsmath,amsfonts,amssymb}
\usepackage[table]{xcolor}
\usepackage{bm,microtype}

\usepackage[pdftex]{graphicx}

\usepackage[T1]{fontenc}
%\usepackage[latin1]{inputenc}
\usepackage{ucs}
\usepackage[utf8x]{inputenc}

\usepackage{lmodern}         % Latin Modern fonts derived from Computer Modern

% Hyperlinks in PDF:
\definecolor{linkcolor}{rgb}{0,0,0.4}
\usepackage{hyperref}
\hypersetup{
    breaklinks=true,
    colorlinks=true,
    linkcolor=linkcolor,
    urlcolor=linkcolor,
    citecolor=black,
    filecolor=black,
    %filecolor=blue,
    pdfmenubar=true,
    pdftoolbar=true,
    bookmarksdepth=3   % Uncomment (and tweak) for PDF bookmarks with more levels than the TOC
    }
%\hyperbaseurl{}   % hyperlinks are relative to this root

\setcounter{tocdepth}{2}  % number chapter, section, subsection

\usepackage[framemethod=TikZ]{mdframed}

% --- begin definitions of admonition environments ---

% --- end of definitions of admonition environments ---

% prevent orhpans and widows
\clubpenalty = 10000
\widowpenalty = 10000

% --- end of standard preamble for documents ---


% insert custom LaTeX commands...

\raggedbottom
\makeindex
\usepackage[totoc]{idxlayout}   % for index in the toc
\usepackage[nottoc]{tocbibind}  % for references/bibliography in the toc

%-------------------- end preamble ----------------------

\begin{document}

% matching end for #ifdef PREAMBLE
% #endif


% ------------------- main content ----------------------



% ----------------- title -------------------------

\thispagestyle{empty}

\begin{center}
{\LARGE\bf
\begin{spacing}{1.25}
First board meeting of Master program in Computational Physics, Mathematics and Life Science 
\end{spacing}
}
\end{center}

% ----------------- author(s) -------------------------

\begin{center}
{\bf Andreas Austeng${}^{1}$} \\ [0mm]
\end{center}


\begin{center}
{\bf Michele Cascella${}^{2}$} \\ [0mm]
\end{center}


\begin{center}
{\bf Marianne Fyhn${}^{3}$} \\ [0mm]
\end{center}


\begin{center}
{\bf Morten Hjorth-Jensen${}^{4}$} \\ [0mm]
\end{center}


\begin{center}
{\bf Hans Petter Langtangen${}^{1, 5}$} \\ [0mm]
\end{center}


\begin{center}
{\bf Anders Malthe-Sørenssen${}^{4}$} \\ [0mm]
\end{center}


\begin{center}
{\bf Knut Mørken${}^{6}$} \\ [0mm]
\end{center}


\begin{center}
{\bf Grete Stavik-Døvle${}^{4}$} \\ [0mm]
\end{center}


\begin{center}
{\bf Joakim Sundnes${}^{1, 5}$} \\ [0mm]
\end{center}


\begin{center}
{\bf Marte Julie Sætra${}^{4}$} \\ [0mm]
\end{center}

\begin{center}
% List of all institutions:
\centerline{{\small ${}^1$Department of Informatics, University of Oslo}}
\centerline{{\small ${}^2$Department of Chemistry, University of Oslo}}
\centerline{{\small ${}^3$Department of Biosciences, University of Oslo}}
\centerline{{\small ${}^4$Department of Physics, University of Oslo}}
\centerline{{\small ${}^5$Simula Research Laboratory}}
\centerline{{\small ${}^6$Department of Mathematics, University of Oslo}}
\end{center}
    
% ----------------- end author(s) -------------------------

% --- begin date ---
\begin{center}
November 4, 2015
\end{center}
% --- end date ---

\vspace{1cm}


% !split
\subsection*{Master program in Computational Physics, Mathematics and Life Science}

% --- begin paragraph admon ---
\paragraph{}

The program is a collaboration between five departments and classical disciplines:

\begin{itemize}
 \item Department of Biosciences

 \item Department of Chemistry

 \item Department of Informatics

 \item Department of Mathematics

 \item Department of Physics
\end{itemize}

\noindent
The program is multidisciplinary and all students who have completed
undergraduate studies in science and engineering, with a sufficient
quantitative background, are eligible.  The language of instruction is
English.
% --- end paragraph admon ---




% !split
\subsection*{Agenda November 4, 2015, 10am-12pm}

% --- begin paragraph admon ---
\paragraph{}
Our main tasks are to discuss and agree on learning goals and competences as well as admission criteria. The deadline for the first \textbf{iteration} to MN-fac is November 10. 

\begin{enumerate}
\item Welcome and presentation of board members and invited guests, 15 min

\item Overview of the material that has been developed and rationale for the program, 20 min

\item Discussion of learning outcomes and competences, see material on \href{{http://mhjensen.github.io/CPMLS/doc/pub/Masterprogram/html/Masterprogram.html}}{github address}, 40 min

\item Discussions and topics to prepare for next meetings, 30 min
\begin{itemize}

  \item Thesis and research directions, preparing documentation and pool of potential supervisors

  \item New members from Department of Mathematics

  \item Integrating the program with NMBU

  \item Exisiting Courses and courses/modules to be developed 

\end{itemize}

\noindent
\item Additional topics
\end{enumerate}

\noindent
% --- end paragraph admon ---



% !split
\subsection*{Some overarching topics}

% --- begin paragraph admon ---
\paragraph{}
\begin{itemize}
\item Numerical simulations of various systems in science are central to our basic understanding of nature and technlogy. UiO is very strong on computational science. A dedicated master program will send strong signals to new students.

\item To build up an activity in computational life science

\item Develop new research directions in computational science with a \textbf{focus on multiscale science and cross-disciplinary research}

\item Strengthen the CSE initiative

\item Some of us have a wider agenda, a new department in Computational Science, along the lines of that developed at \href{{http://msutoday.msu.edu/news/2015/new-department-advances-computational-science-research-education/}}{Michigan State University}
\end{itemize}

\noindent
% --- end paragraph admon ---





% !split
\subsection*{Integrating with NMBU}

% --- begin paragraph admon ---
\paragraph{Two paths:}

\begin{itemize}
\item A common program, owned by UiO and NMBU. Students apply locally and register their credits locally but can have supervisors and take courses from both universities. The degree is awarded locally. This is most likely the simplest path. 

\item A cotutelle program, with the same modalities as above but common degrees. This is more difficult and UiO has not been very keen about cotutelle agreements at the MSc level or PhD level.  
\end{itemize}

\noindent
% --- end paragraph admon ---




% !split
\subsection*{Thesis directions (and need of new board members)}

% --- begin paragraph admon ---
\paragraph{}
\begin{itemize}
\item Computational mathematics

\item Computational mechanics and fluid mechanics  (NEED people here)

\item Computational chemistry

\item Computational physics

\item Computational materials science

\item Computational life science

\item Image analysis and signal processing

\item Computational finance and statistics   (NEED people here)
\end{itemize}

\noindent
% --- end paragraph admon ---



% !split
\subsection*{Tasks}

% --- begin paragraph admon ---
\paragraph{}
\begin{itemize}
\item Identify courses and courses that can be modularized and integrated in the program

\item Develop a pool of MSc thesis projects  

\item Identify a pool of potential MSc supervisors

\item Identify eventual new board members
\end{itemize}

\noindent
% --- end paragraph admon ---








% ------------------- end of main content ---------------

% #ifdef PREAMBLE
\cleardoublepage\phantomsection  % trick to get correct link to Index
\printindex

\end{document}
% #endif

