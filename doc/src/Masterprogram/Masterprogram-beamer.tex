
% LaTeX Beamer file automatically generated from DocOnce
% https://github.com/hplgit/doconce

%-------------------- begin beamer-specific preamble ----------------------

\documentclass{beamer}

\usetheme{red_plain}
\usecolortheme{default}

% turn off the almost invisible, yet disturbing, navigation symbols:
\setbeamertemplate{navigation symbols}{}

% Examples on customization:
%\usecolortheme[named=RawSienna]{structure}
%\usetheme[height=7mm]{Rochester}
%\setbeamerfont{frametitle}{family=\rmfamily,shape=\itshape}
%\setbeamertemplate{items}[ball]
%\setbeamertemplate{blocks}[rounded][shadow=true]
%\useoutertheme{infolines}
%
%\usefonttheme{}
%\useinntertheme{}
%
%\setbeameroption{show notes}
%\setbeameroption{show notes on second screen=right}

% fine for B/W printing:
%\usecolortheme{seahorse}

\usepackage{pgf}
\usepackage{graphicx}
\usepackage{epsfig}
\usepackage{relsize}

\usepackage{fancybox}  % make sure fancybox is loaded before fancyvrb

\usepackage{fancyvrb}
%\usepackage{minted} % requires pygments and latex -shell-escape filename
%\usepackage{anslistings}
%\usepackage{listingsutf8}

\usepackage{amsmath,amssymb,bm}
%\usepackage[latin1]{inputenc}
\usepackage[T1]{fontenc}
\usepackage[utf8]{inputenc}
\usepackage{colortbl}
\usepackage[english]{babel}
\usepackage{tikz}
\usepackage{framed}
% Use some nice templates
\beamertemplatetransparentcovereddynamic

% --- begin table of contents based on sections ---
% Delete this, if you do not want the table of contents to pop up at
% the beginning of each section:
% (Only section headings can enter the table of contents in Beamer
% slides generated from DocOnce source, while subsections are used
% for the title in ordinary slides.)
\AtBeginSection[]
{
  \begin{frame}<beamer>[plain]
  \frametitle{}
  %\frametitle{Outline}
  \tableofcontents[currentsection]
  \end{frame}
}
% --- end table of contents based on sections ---

% If you wish to uncover everything in a step-wise fashion, uncomment
% the following command:

%\beamerdefaultoverlayspecification{<+->}

\newcommand{\shortinlinecomment}[3]{\note{\textbf{#1}: #2}}
\newcommand{\longinlinecomment}[3]{\shortinlinecomment{#1}{#2}{#3}}

\definecolor{linkcolor}{rgb}{0,0,0.4}
\hypersetup{
    colorlinks=true,
    linkcolor=linkcolor,
    urlcolor=linkcolor,
    pdfmenubar=true,
    pdftoolbar=true,
    bookmarksdepth=3
    }
\setlength{\parskip}{0pt}  % {1em}

\newenvironment{doconceexercise}{}{}
\newcounter{doconceexercisecounter}
\newenvironment{doconce:movie}{}{}
\newcounter{doconce:movie:counter}

\newcommand{\subex}[1]{\noindent\textbf{#1}}  % for subexercises: a), b), etc

%-------------------- end beamer-specific preamble ----------------------

% Add user's preamble




% insert custom LaTeX commands...

\raggedbottom
\makeindex

%-------------------- end preamble ----------------------

\begin{document}

% matching end for #ifdef PREAMBLE
% #endif



% ------------------- main content ----------------------



% ----------------- title -------------------------

\title{Master program in Computational Physics, Mathematics and Life Science}

% ----------------- author(s) -------------------------

\author{Andreas Austeng\inst{1}
\and
Michele Cascella\inst{2}
\and
Marianne Fyhn\inst{3}
\and
Morten Hjorth-Jensen\inst{4}
\and
Hans Petter Langtangen\inst{1,5}
\and
Anders Malthe-Sørenssen\inst{4}
\and
Knut Mørken\inst{6}
\and
Joakim Sundnes\inst{1,5}}
\institute{Department of Informatics, University of Oslo\inst{1}
\and
Department of Chemistry, University of Oslo\inst{2}
\and
Department of Biosciences, University of Oslo\inst{3}
\and
Department of Physics, University of Oslo\inst{4}
\and
Simula Research Laboratory\inst{5}
\and
Department of Mathematics, University of Oslo\inst{6}}
% ----------------- end author(s) -------------------------

\date{October 2015
% <optional titlepage figure>
% <optional copyright>
}

\begin{frame}[plain,fragile]
\titlepage
\end{frame}

\begin{frame}[plain,fragile]
\frametitle{Master program in Computational Physics, Mathematics and Life Science}

\begin{block}{}
We propose a new Master of Science program at the Faculty of Mathematics and Natural Sciences of the University of Oslo. This program is called  \textbf{Computational Physics, Mathematics and Life Science}, with acronym  \textbf{CPMLS} 


The program is a collaboration between five departments and classical disciplines:

\begin{itemize}
 \item Department of Biosciences

 \item Department of Chemistry

 \item Department of Informatics

 \item Department of Mathematics

 \item Department of Physics
\end{itemize}

\noindent
The program will be administrated by the Department of Physics. 
The program is based on the highly successful Computational Physics direction under the present Master program
in physics at the University of Oslo.

The program is multidisciplinary and all students who have completed
undergraduate studies in science and engineering, with a sufficient
quantitative background, are eligible.  The language of instruction is
English.  

% add about UMB

\end{block}
\end{frame}

\begin{frame}[plain,fragile]
\frametitle{Strategic importance}

The program will educate the next generation of cross-disciplinary
science students with the knowledge, skills, and values needed to pose
and solve current and new scientific, technological and societal
challenges. The program will lay the foundation for cross-disciplinary
educational, research and innovation activities at the Faculty. The
program will contribute to building a common cross-disciplinary
approach to the key strategic initiatives at the Faculty: Energy,
Materials, Life Science, and Enabling Technologies.

A particular strength of physics students is their ability to pose and
solve problems that combine physical insights with mathematical tools
and now also computational skills. This provides a unique combination
of applied and theoretical knowledge and skills. These features are invaluable 
for the development of multi-disciplinary educational and research programs. 
In this program we build on and
refine this philosophy.  The main focus is not to educate computer
specialists, but to educate students with a solid understanding in basic science
as well as an integrated knowledge on how  to use 
essential methods from computational science. This requires an
education that covers both the specific disciplines like physics, biology,
geoscience, mathematics etc with a strong background in computational science.
\end{frame}

\begin{frame}[plain,fragile]
\frametitle{Scientific and educational motivation}

\begin{block}{Applications of simulation }
Numerical simulations of various systems in science are central to our
basic understanding of nature and technlogy.
The increase in computational power,
improved algorithms for solving problems in science as well as access
to high-performance facilities, allow researchers nowadays to study
complicated systems across many length and energy scales. Applications
span from studying quantum physical systems in nanotechnology and the
characteristics of new materials or subamotic physics at its smallest
length scale, to simulating galaxies and the evolution of the universe.
In between, simulations are key to understanding
cancer treatment and how the brain works,
predicting climate changes and this week's weather,
simulating natural disasters, semi-conductor devices,
quantum computers, as well as assessing risk in the insurance and
financial industry. These are just a few topics
already well covered at the University of Oslo and that can be
topics for coming thesis projects as well as research directions.
\end{block}

\begin{block}{Job market }
A large number of the candidates from the four involved departments
get jobs where numerical simulations are central and essential. The proposed
program will raise the educational quality in this area, because
our candidates need a broader understanding of the possibilities
and limitations of computation-based problem solving.
\end{block}
\end{frame}

\begin{frame}[plain,fragile]
\frametitle{Multiscale modeling is the big open research question}

\begin{block}{}
Today's problems, unlike traditional
science and engineering, involve complex systems with many distinct
physical processes. The wide open research topic of this century, both
in industry and at universities, is how to effectively couple
processes across different length and energy scales. Progress will
rely on a multi-disciplinary approach and therefore a need for
a multi-disciplinary educational program.
\end{block}

\begin{block}{}
The proposed program will foster candidates with the right
multi-disciplinary background and comutational thinking for
understanding today's simulation technology and its challenges.
\end{block}
\end{frame}

\begin{frame}[plain,fragile]
\frametitle{The new program combines old and new initiatives}

\begin{block}{}

This program builds on the strengths and successes of two existing master's
programs in Computational Physics (at the Dept.~of Physics) and
Applied Mathematics and Mechanics (at the Dept.~of Mathematics).

A new master's program in Computational Life must anyway be developed to
meet coming needs of the scientific community. If successful, it will
position the University of Oslo as the leading institution
nationally in computational life science. We belive this new program is
best developed in close collaboration with already successful
computational science programs.

The program in Computational Life Sciences will have a strong
link to the Norwegian University of Life Sciences. Further links to
NTNU will be developed.
\end{block}
\end{frame}

\begin{frame}[plain,fragile]
\frametitle{Computational Physics at UiO has been a great success}

\begin{block}{}

This initiative has its roots in the highly successful direction called \href{{http://www.uio.no/english/studies/programmes/physics-master/programme-options/computational/index.html}}{Computational Physics}
under the Master program in Physics at the University of Oslo.

This program has educated almost 60 Master of Science students during
the last ten years.  Over 50\% of these students have continued with
PhD studies in Physics, Chemistry, Mathematics and now recently
Biology connected with the CINPLA project.

\end{block}
\end{frame}

\begin{frame}[plain,fragile]
\frametitle{The new program will also host the CSE project}

\begin{block}{}

The new proposed program will also take a leading responsibility in further
developments of the highly successful \href{{http://www.mn.uio.no/english/about/collaboration/cse/}}{Computing in Science Education} initiative at UiO.  Master of science thesis projects linked up to the CSE project will be offered.

If the program becomes successful, it will naturally lead to
new cross-disciplinary research and a need for a new department
in computational science.
\end{block}
\end{frame}

\begin{frame}[plain,fragile]
\frametitle{Computing competence}

\begin{block}{}
Computing means solving scientific problems using computers. It covers
numerical as well as symbolic computing. Computing is also about
developing an understanding of the scientific process by enhancing the
algorithmic thinking when solving problems.  Computing competence has
always been a central part of the science and engineering
education. 

Computing competence is about

\begin{itemize}
\item derivation, verification, and implementation of algorithms

\item understanding what can go wrong with algorithms

\item overview of important, known algorithms

\item understanding how algorithms are used to solve mathematical problems

\item reproducible science and ethics

\item algorithmic thinking for gaining deeper insights about scientific problems
\end{itemize}

\noindent
\end{block}
\end{frame}

\begin{frame}[plain,fragile]
\frametitle{Key elememts in computing competence}

\begin{block}{}
The power of the scientific method lies in identifying a given problem
as a special case of an abstract class of problems, identifying
general solution methods for this class of problems, and applying a
general method to the specific problem (applying means, in the case of
computing, calculations by pen and paper, symbolic computing, or
numerical computing by ready-made and/or self-written software). This
generic view on problems and methods is particularly important for
understanding how to apply available, generic software to solve a
particular problem.  


Computing competence represents a central element
in scientific problem solving, from basic education and research to
essentially almost all advanced problems in modern
societies. Computing competence is simply central to further
progress. It enlarges the body of tools available to students and
scientists beyond classical tools and allows for a more generic
handling of problems. Focusing on algorithmic aspects results in
deeper insights about scientific problems.

Today's projects in science and industry tend to involve larger teams. Tools for reliable collaboration must therefore be mastered (e.g., version control systems, automated computer experiments for reproducibility, software and method documentation).
\end{block}
\end{frame}

\begin{frame}[plain,fragile]
\frametitle{Learning outcomes}

\begin{block}{}
Students of this program learn to use the computer as a laboratory for solving problems in science and engineering. The program offers exciting thesis projects from many disciplines: biology and life science, chemistry, mathematics, informatics, physics, geophysics, mechanics, geology, computational finance, digital signal processing and image analysis  – the students choose their  field according to their own interests. 

A Master’s degree from this program gives the student   a methodical training in planning, conducting, and reporting large research projects, often together with other students and university teachers. 
Projects usually emphasise finding practical solutions,
developing an intuitive understanding of the science and the
scientific methods needed to solve complicated problems, use of many
tools, and not least developing your own creativity and independent
thinking. The thesis work is a scientific project where the students learn to
tackle a scientific problem in a professional manner.
\end{block}
\end{frame}

\begin{frame}[plain,fragile]
\frametitle{Specific skills}

\begin{block}{}
The students learns to understand and develop insights in high-level scientific problems, involving a fundamental understanding of the methods and tools which are necessary and to present these results orally and in written form as scientific reports. In addition, students learn to get

\begin{itemize}
\item Deep knowledge of the most fundamental algorithms involved, how optimizatize these  and statistical uncertainty quantification.

\item Overview of advanced algorithms and how they can be accessed in available software.

\item Knowledge of high-performance computing elements: memory usage, vectorized and parallel algorithms.

\item Understanding of approximation errors.

\item Knowledge of at least one computer algebra system and how it is applied to perform classical mathematics (calculus, linear algebra, differential equations - with verification).

\item Extensive experience with programming in a high-level language (MATLAB, Python, R). Experience with programming in a compiled language (Fortran, C, C++).

\item Experience with implementing and applying numerical algorithms in reusable software that acknowledges the generic nature of the mathematical algorithms.

\item Experience with debugging software, e.g., as part of implementing comprehensive tests.

\item Experience with programming of testing procedures.

\item Experience with different visualization techniques for different types of computed data.

\item Experience with presenting computed results in scientific reports and oral presentations.

\item Critical evaluation of results and errors

\item Be able to develop software and algorithms for solving complicated scientific problems independently and/or in collaboration with other students.
\end{itemize}

\noindent
\end{block}
\end{frame}

\begin{frame}[plain,fragile]
\frametitle{Structure and courses}

\begin{block}{}
The table here is an example of a suggested path for a Master of Science project,
with course work the first year and thesis work the last year.


{\footnotesize
\begin{tabular}{llll}
\hline
\multicolumn{1}{l}{  } & \multicolumn{1}{l}{ 10 ECTS } & \multicolumn{1}{l}{ 10 ECTS } & \multicolumn{1}{l}{ 10 ECTS } \\
\hline
4th semester & Master thesis  & Master Thesis  & Master Thesis  \\
\hline
3rd semester & Master thesis  & Master Thesis  & Master Thesis  \\
\hline
2nd semester & Master courses & Master courses & Master courses \\
\hline
1st semester & Master courses & Master courses & Master courses \\
\hline
\end{tabular}
}

\noindent
The program is very flexible in its structure and students may opt for starting with their thesis
work from the first semester and scatter the respective course load across all four semesters.

Depending on interests and specializations, there are many courses on computational science which can make
up the required curriculum of course work. Furthermore, courses may be broken up in smaller modules,
avoding thereby the limitation of 10 ECTS per course only. Some of these courses are listed below.
\end{block}
\end{frame}

\begin{frame}[plain,fragile]
\frametitle{Structure}

\begin{block}{}
Here follows a list of suggested courses that students may include in their required course load.

\begin{itemize}
\item \href{{http://www.uio.no/studier/emner/matnat/fys/FYS4150/index-eng.html}}{FYS4150 Computational Physics I}

\item \href{{http://www.uio.no/studier/emner/matnat/fys/FYS4411/}}{FYS4411 Computational Physics II}

\item \href{{http://www.uio.no/studier/emner/matnat/fys/FYS4460/}}{FYS4460 Computational Physics III}

\item \href{{http://www.uio.no/studier/emner/matnat/ifi/INF5620/index-eng.html}}{INF5620 Numerical Methods for Partial Differential Equations}

\item \href{{http://www.uio.no/studier/emner/matnat/ifi/INF5631/index-eng.html}}{INF5631 Project on Numerical Methods for Partial Differential Equations}

\item \href{{http://www.nmbu.no/course/FYS388}}{FYS388 Computational Neuroscience}

\item \href{{http://www.uio.no/studier/emner/matnat/math/STK4520/index-eng.html}}{STK4520 Laboratory for Finance and Insurance Mathematics}

\item \href{{http://www.uio.no/studier/emner/matnat/math/STK4021/index-eng.html}}{STK4021 Applied Bayesian Analysis and Numerical Methods}

\item \href{{http://www.uio.no/studier/emner/matnat/math/MAT-INF4130/index-eng.html}}{MAT-INF4130  Numerical Linear Algebra}

\item \href{{http://www.uio.no/studier/emner/matnat/math/MAT-INF4110/index.html}}{MAT-INF4110 Mathematical Optimization}

\item \href{{http://www.uio.no/studier/emner/sv/oekonomi/ECON4240/index.html}}{ECON4240 Equilibrium, welfare and information}

\item \href{{http://www.uio.no/studier/emner/matnat/math/MEK4470/index-eng.html}}{MEK4470  Computational Fluid Mechanics}

\item \href{{http://www.uio.no/studier/emner/matnat/math/MEK4250/index-eng.html}}{MEK4250 Finite Element Methods in Computational Mechanics}
\end{itemize}

\noindent
The program plans to develop other courses in computational science and its applications, ranging from life science to materials science.
\end{block}
\end{frame}

\begin{frame}[plain,fragile]
\frametitle{Admission}

\begin{block}{}
The following higher education entrance qualifications are needed

\begin{itemize}
\item A completed bachelor's degree (undergraduate) comparable to a Norwegian bachelor's degree in one of the following disciplines
\begin{enumerate}

 \item Biology, molecular biology, biochemistry  or any life science degree

 \item Physics, astrophysics, astronomy, geophysics and meteorology

 \item Mathematics, mechanics, statistics and computational mathematics

 \item Computer science and electronics

 \item Chemistry

 \item Materials Science and nanotechnology

 \item Any undergraduate degree in engineering

 \item Mathematical finance and economy

\end{enumerate}

\noindent
\item The language of instruction is English. An internationally recognised English language proficiency test is required.
\end{itemize}

\noindent
\end{block}
\end{frame}

\begin{frame}[plain,fragile]
\frametitle{The program opens up for flexible backgrounds}

\begin{block}{}
While discipline-based master's programs tend to introduce very strict
requirements to courses, we believe in adapting a computational thesis
topic to the student's background, thereby opening up for
students with a wide range of bachelor's degrees.
A very heterogeneous student community is thought to be a strength and
unique feature of the new program.
\end{block}
\end{frame}

\begin{frame}[plain,fragile]
\frametitle{Study abroad and international collaborators}

\begin{block}{}

Students at the University of Oslo may choose to take part of
their degrees at a university abroad.

Students in this program have a number of interesting international
exchange possibilities. The involved researchers have extensive
collaborations with other researchers worldwide. These exchange
possibility range from top universities in the USA, Asia and Europe as
well as leading National Laboratories in the USA.  
% Students may select Ato take all or part of their degree abroad.
\end{block}
\end{frame}

\begin{frame}[plain,fragile]
\frametitle{Career prospects}

\begin{block}{}
Candidates who are capable of modeling and understanding complicated
systems in natural science, are in short supply in society.  The
computational methods and approaches to scientific problems students learn
when working on their thesis projects are very similar to the methods
they will use in later stages of their careers.  To handle large
numerical projects demands structured thinking and good analytical
skills and a thorough understanding of the problems to be solved. This
knowledge makes the students unique on the labor market.

Career opportunities are many, from research institutes, universities
and university colleges and a multitude of companies. Examples
include IBM, Hydro, Statoil, and Telenor.  The program gives an
excellent background for further studies, with a PhD as one possible
goal.

The program has also a strong international element which allows students to
gain important experience from international collaborations in
science, with the opportunity to spend parts of the time spent on 
thesis work at research institutions abroad.
\end{block}
\end{frame}

\end{document}
