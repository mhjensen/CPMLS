%%
%% Automatically generated file from DocOnce source
%% (https://github.com/hplgit/doconce/)
%%
%%


%-------------------- begin preamble ----------------------

\documentclass[%
twoside,                 % oneside: electronic viewing, twoside: printing
final,                   % or draft (marks overfull hboxes, figures with paths)
10pt]{article}

\listfiles               % print all files needed to compile this document

\usepackage{relsize,makeidx,color,setspace,amsmath,amsfonts}
\usepackage[table]{xcolor}
\usepackage{bm,microtype}

\usepackage[T1]{fontenc}
%\usepackage[latin1]{inputenc}
\usepackage{ucs}
\usepackage[utf8x]{inputenc}

\usepackage{lmodern}         % Latin Modern fonts derived from Computer Modern

% Hyperlinks in PDF:
\definecolor{linkcolor}{rgb}{0,0,0.4}
\usepackage{hyperref}
\hypersetup{
    breaklinks=true,
    colorlinks=true,
    linkcolor=linkcolor,
    urlcolor=linkcolor,
    citecolor=black,
    filecolor=black,
    %filecolor=blue,
    pdfmenubar=true,
    pdftoolbar=true,
    bookmarksdepth=3   % Uncomment (and tweak) for PDF bookmarks with more levels than the TOC
    }
%\hyperbaseurl{}   % hyperlinks are relative to this root

\setcounter{tocdepth}{2}  % number chapter, section, subsection

\usepackage[framemethod=TikZ]{mdframed}

% --- begin definitions of admonition environments ---

% --- end of definitions of admonition environments ---

% prevent orhpans and widows
\clubpenalty = 10000
\widowpenalty = 10000

% --- end of standard preamble for documents ---


% insert custom LaTeX commands...

\raggedbottom
\makeindex

%-------------------- end preamble ----------------------

\begin{document}

% endif for #ifdef PREAMBLE


% ------------------- main content ----------------------



% ----------------- title -------------------------

\thispagestyle{empty}

\begin{center}
{\LARGE\bf
\begin{spacing}{1.25}
Master program in Computational Physics, Mathematics and Life Science
\end{spacing}
}
\end{center}

% ----------------- author(s) -------------------------

\begin{center}
{\bf Marianne Fyhn${}^{1}$} \\ [0mm]
\end{center}

    
\begin{center}
{\bf Morten Hjorth-Jensen${}^{2}$} \\ [0mm]
\end{center}

    
\begin{center}
{\bf Hans Petter Langtangen${}^{3, 4}$} \\ [0mm]
\end{center}

    
\begin{center}
{\bf Anders Malthe-Sørenssen${}^{2}$} \\ [0mm]
\end{center}

    
\begin{center}
{\bf Knut Mørken${}^{5}$} \\ [0mm]
\end{center}

    \begin{center}
% List of all institutions:
\centerline{{\small ${}^1$Department of Biosciences, University of Oslo}}
\centerline{{\small ${}^2$Department of Physics, University of Oslo}}
\centerline{{\small ${}^3$Department of Informatics, University of Oslo}}
\centerline{{\small ${}^4$Simula Research Laboratory}}
\centerline{{\small ${}^5$Department of Mathematics, University of Oslo}}
\end{center}
    
% ----------------- end author(s) -------------------------

\begin{center} % date
May 6 2015
\end{center}

\vspace{1cm}


% !split
\subsection*{Master program in Computational Physics, Mathematics and Life Science (CPMLS)}

% --- begin paragraph admon ---
\paragraph{}
We would like to propose a new Master of Science program at the School of Mathematics and Natural Sciences of the University of Oslo. This program is called  \textbf{Computational Physics, Mathematics and Life Science}, with acronym  \textbf{CPMLS}. 

The program is a collaboration between four departments and four main disciplines,
\begin{itemize}
\item Department of Biosciences 

\item Department of Informatics

\item Department of Mathematics 

\item Department of Physics 
\end{itemize}

\noindent
The Department of Physics will be the organizational unit where the program resides. 
The program is based on the highly successful Computational Physics direction under the present Master program
in physics at the University of Oslo. 

The program is multidisciplinary and all students who have completed undergraduate studies in science and enginereeing are eligible.  The language of instruction is English.
% --- end paragraph admon ---







% !split
\subsection*{Overarching aims and strategic perspective}

% --- begin paragraph admon ---
\paragraph{}


Numerical simulations of various systems in science are central to our
basic understanding of nature.  The increase in computational power,
improved algorithms for solving problems in science as well as access
to high-performance facilities, allows researchers nowadays to study
complicated systems across many length and energy scales. Applications
span from studying quantum physical systems in nanotechnology and the
characteristics of new materials or subamotic physics at its smallest
length scale, to simulating cancer treatment, how the brain works,
modeling climate and weather, oil flow through various rock strata,
simulating natural disasters, semi-conductor technology, simulating
quantum computers, financial engineering etc, just to mention a few
possible directions of research and study for a thesis.
% --- end paragraph admon ---




% !split
\subsection*{Overarching aims and strategic perspective}

% --- begin paragraph admon ---
\paragraph{}

Simulations that couple multiple physical phenomena are
as old as simulations themselves. However, such
simulations deserve fresh assessment, in light of steadily
increasing computational capability and greater aspirations
for simulation in domains of scientific prediction, engineering
design, and policy making.

Of particular importance for technological advances in fields as varied as materials science and life science, or subatomic physics, is the capability to study and model physical phenomena across scales, what is normally called multiscale science.
Today's problems, unlike traditional science and engineering, do not  involve physical processes covered by the single traditional discipline of say physics or the associated mathematics. Complex and real systems encountered  in virtually all applications of interest involve many distinct physical processes. 
The issue of coupling models of different events at different scales (length or energy) and governed by different physical laws is largely wide open and represents an enormously challenging area for future research and technological advances.  Tackling problems from multiscale science requires multi disciplinarity as well.
% --- end paragraph admon ---





% !split
\subsection*{Overarching aims and strategic perspective}

% --- begin paragraph admon ---
\paragraph{}

This program aims  at preparing the next generation of scientists enabling them to tackle the many challenges posed by understanding physical processes across many scales. 

Since this requires cooperations between many disciplines, a program
which involves several departments is needed. Furthermore, this
program aims at developing a master's program in Computational Life
Science in order to meet the coming needs of the scientific community. If successful, it will
position the University of Oslo as the leading institution
nationally in computational life science and computational science in general. 

In addition to collaborations across disciplines at the University of Oslo, 
this program will also link to the Norwegian University of Life Sciences 
in order to build a strong basis in computational science via collaborations with other universities.
% --- end paragraph admon ---




% !split
\subsection*{Overarching aims and strategic perspective}

% --- begin paragraph admon ---
\paragraph{}

The initiative has its roots in the highly successful direction called \href{{http://www.uio.no/english/studies/programmes/physics-master/programme-options/computational/index.html}}{Computational Physics}
under the Master program in Physics at the University of Oslo.

This program has educated almost 60 Master of Science students during the last ten years.
Over 50\% of these students have continued with PhD studies in Physics, Chemistry, Mathematics and
now recently Biology connected with the CINPLA projects. 

Seen the popularity and versatility of the Computational Physics program and the abovementioned coming needs for multi-disciplinarity in science, we would thus like to propose a new Master of Science program which includes computational physics, computational mathematics and computational life science. As stated above, it will be a collaboration between four departments, with the admistrative responsibility residing with the department of Physics. 

This program will also take a leading responsibility in further
developments of the highly successful \href{{http://www.mn.uio.no/english/about/collaboration/cse/}}{Computing in Science Education} initiative at UiO. 

\begin{itemize}
\item Should we say something about a new department in computational science?
\end{itemize}

\noindent
% --- end paragraph admon ---








% !split
\subsection*{Learning outcomes}

% --- begin paragraph admon ---
\paragraph{}

Students of this program learn to use the computer as a laboratory for solving problems in science and engineering. The program offers exciting thesis projects from several disciplines; biology and life science, chemistry, mathematics, informatics, physics, geophysics, mechanics, geology  – you choose your  field according to your own interests.

In addition to this, a Master’s degree from this program gives you a
methodical training in planning and carrying out large research
projects, often together with other students and university
teachers. Projects usually emphasise finding practical solutions,
developing an intuitive understanding of the science and the
scientific methods needed to solve complicated problems, use of many
tools, and not least developing your own creativity and independent
thinking. Your thesis work is a scientific  project and during your work you will
learn to plan and conduct large-scale projects logically and efficiently, as well as
to report and present results in a professional manner.
% --- end paragraph admon ---






% !split
\subsection*{Structure and courses}

% --- begin paragraph admon ---
\paragraph{}
The table here is an example of a suggested path for a Master of Science project,
with course work the first year and thesis work the last year. 

\begin{quote}
\begin{tabular}{cccc}
\hline
\multicolumn{1}{c}{ 4th semester } & \multicolumn{1}{c}{ Master thesis } & \multicolumn{1}{c}{ Master Thesis } & \multicolumn{1}{c}{ Master Thesis } \\
\hline
3rd semester & Master thesis  & Master Thesis  & Master Thesis  \\
\hline
2nd semester & Master courses & Master courses & Master courses \\
\hline
1st semester & Master courses & Master courses & Master courses \\
\hline
             & 10 ECTS        & 10 ECTS        & 10 ECTS        \\
\hline
\end{tabular}
\end{quote}

\noindent
The program is very flexible in its structure and students may opt for starting with their thesis 
work from the first semester and scatter the respective course load across all four semesters.

Depending on interests and specializations, there are many courses on computational science which can make 
up the required curriculum of course work. Furthermore, courses may be broken up in smaller modules,
avoding thereby the limitation of 10 ECTS per course only. Some of these courses are listed below.
% --- end paragraph admon ---




% !split
\subsection*{Structure}

% --- begin paragraph admon ---
\paragraph{}
Here follows a list of suggested courses that students may include in their required course load.
\begin{itemize}
\item \href{{http://www.uio.no/studier/emner/matnat/fys/FYS4150/index-eng.html}}{FYS4150 Computational Physics I}

\item \href{{http://www.uio.no/studier/emner/matnat/fys/FYS4411/}}{FYS4411 Computational Physics II}

\item \href{{http://www.uio.no/studier/emner/matnat/fys/FYS4460/}}{FYS4460 Computational Physics III}

\item \href{{http://www.uio.no/studier/emner/matnat/ifi/INF5620/index-eng.html}}{INF5620 Numerical Methods for Partial Differential Equations}

\item \href{{http://www.nmbu.no/course/FYS388}}{FYS388 Computational Neuroscience}

\item \href{{http://www.uio.no/studier/emner/matnat/math/STK4520/index-eng.html}}{STK4520 Laboratory for Finance and Insurance Mathematics}

\item \href{{http://www.uio.no/studier/emner/matnat/math/STK4021/index-eng.html}}{STK4021 Applied Bayesian Analysis and Numerical Methods}

\item \href{{http://www.uio.no/studier/emner/matnat/math/MAT-INF4130/index-eng.html}}{MAT-INF4130  Numerical Linear Algebra}

\item \href{{http://www.uio.no/studier/emner/matnat/math/MAT-INF4110/index.html}}{MAT-INF4110 Mathematical Optimization}

\item \href{{http://www.uio.no/studier/emner/sv/oekonomi/ECON4240/index.html}}{ECON4240 Equilibrium, welfare and information}

\item \href{{http://www.uio.no/studier/emner/matnat/math/MEK4470/index-eng.html}}{MEK4470  Computational Fluid Mechanics}

\item \href{{http://www.uio.no/studier/emner/matnat/math/MEK4250/index-eng.html}}{MEK4250 Finite Element Methods in Computational Mechanics}
\end{itemize}

\noindent
The program, if approved, plans to develop other courses in computational science and its applications, ranging from life science to materials science.
% --- end paragraph admon ---







% !split
\subsection*{Admission}

% --- begin paragraph admon ---
\paragraph{}
The following higher education entrance qualifications are needed
\begin{itemize}
\item A completed bachelor's degree (undergraduate) comparable to a Norwegian bachelor's degree in one of the following disciplines
\begin{enumerate}

 \item Biology, molecular biology, biochemistry  or any life science degree

 \item Physics, astrophysics, astronomy, geophysics and meteorology

 \item Mathematics, mechanics, statistics and computational mathematics

 \item Computer science and electronics

 \item Chemistry

 \item Materials Science and nanotechnology

 \item Any undergraduate degree in engineering

 \item Mathematical finance and economy

\end{enumerate}

\noindent
\item The language of instruction is English. An internationally recognised English language proficiency test is required.
\end{itemize}

\noindent
% --- end paragraph admon ---




% !split
\subsection*{Study abroad and international collaborators}

% --- begin paragraph admon ---
\paragraph{}

As a student at the University of Oslo you may choose to take part of
your degree at a university abroad. 

Students in this program have a number of interesting international
exchange possibilities. The involved researchers have extensive
collaborations with other researchers worldwide. These exchange
possibility range from top universities in the USA, Asia and Europe as
well as leading National Laboratories in the USA.  Students may select
to take all or part of their degree abroad.
% --- end paragraph admon ---





% !split 
\subsection*{Career prospects}

% --- begin paragraph admon ---
\paragraph{}
Candidates who are capable
of modeling and understanding complicated systems in natural science,
are in short supply in society.  The methods and approaches to
scientific problems you learn when working on your thesis project are
very similar to the methods you will use in later stages of your
career.  To handle large numerical projects demands structured
thinking and good analytical skills and a thorough understanding of
the problems to be solved. This makes you unique on the labor market.

Career opportunities are many, from research institutes, universities
and university colleges and a multitude number of companies. Examples
like IBM, Hydro, Statoil and Telenor.  The program gives you an
excellent background for further studies, with a PhD as an eventual
goal. One of two students with a Master of Science from this program
choose to continue with PhD studies.

The program has also a strong international element which allows you
gain important experiences from international collaborations in
science, with the opportunity to spend parts of the time spent on your
thesis work at research institutions abroad.

Most of our students find employment as researchers at universities,
university colleges, different research centers or in industry.
% --- end paragraph admon ---





% ------------------- end of main content ---------------


\printindex

\end{document}

